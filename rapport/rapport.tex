\documentclass[10pt,a4paper]{article}
\usepackage[utf8]{inputenc}
\usepackage[francais]{babel}
\usepackage[T1]{fontenc}
\author{Ali CHERIFI}
\title{Rapport de stage de licence\\Résumé vidéo et vidéo 3D anaglyphe et side-by-side}
\begin{document}
\maketitle
\newpage
\tableofcontents
\section{Introduction}

\section{Laboratoire d'accueil}



\section{Etat de l'art et matériel existant}
\subsection{Résumé vidéo}

\subsection{Anaglyphe et side-by-side}
Actuellement, le vidéo 3D par anaglyphe propose deux algorithmes majeurs, un algorithme simple et l'algorithme de Dubois.
Le premier algorithme consiste à séparer et extraire les canaux RGB d'une image 2D et d'en faire une image 3D. Le principe réside dans le filtrage des couleurs par l'oeil devant lequel se trouve un filtre.
Chaque oeil ayant un filtre différent, il ne perçoit que les couleurs que le filtre laise passer. Actuellement, le filtre le plus répandu est le filtre rouge/cyan.
Il suffit donc dans le cas de cet algortihme, d'extraire le canal rouge de l'image et d'en génerer un nouvelle. Ensuite il faut à partir de l'image source créer une nouvelle image à partir du mélange des canaux bleu et vert. On obtient ainsi deux images, une cyan et l'autre rouge. Il faut ensuites les superposer en leur appliquant un décalage.


\section{Cahier des charges et architecture}
\section{Implémentation}
\section{Phase de test}

\section{Conclusion}





Besoins fonctionnels : 

S'affiche sous forme de fenêtre.
Choisir fichier(s)
Préciser si la vidéo est décodable par le logiciel ou non. Si non, précisez les formats acceptés.
choix du mode de traitement/algo(anaglyphe choisir entre algo classique et dubois)
Effectuer le traitement à l'aide d'un bouton
Montrer l'avancement du traitement.
Sauvegarder le fichier sous le nom que l'on veut.
ouvrir fichier en fin de traitement ou son emplacement.
prévisualiser la vidéo séléctionnée

Non fonctionnels :
simple
léger
sans bugs.
Au pire tps = 1/1
Aide disponible 

\end{document}
