\documentclass[10pt,a4paper]{article}
\usepackage[utf8]{inputenc}
\usepackage[francais]{babel}
\usepackage[T1]{fontenc}
\author{Ali CHERIFI}
\title{Rapport de stage de licence\\Résumé vidéo et vidéo 3D anaglyphe et side-by-side}
\begin{document}
\maketitle
\newpage
\tableofcontents
\newpage
\section{Introduction}

\section{Laboratoire d'accueil}



\section{Etat de l'art et matériel existant}
\subsection{Résumé vidéo}
Le résumé vidéo peut aussi être appelé "Movie Barcode". Il permet de faire ressortir la couleur générale d'un film ou d'une vidéo. En récupérant la bande du milieu de chaque image et en concatenant tout cela pour
tout le film, on obtient une image très similaire à un code barre représentant l'ambiance générale de la vidéo.
Il existe un logiciel disponible réalisant ce traitement nommé "Movie Barcode Generator". Le logiciel est écrit en C\# et ses sources sont disponibles à tous. L'auteur du logiciel accepte la vente des
résultats produit par son logiciel.

\subsection{Anaglyphe et side-by-side}
Actuellement, le vidéo 3D par anaglyphe propose deux méthodes majeures, un algorithme simple et la méthode de Dubois.
Le premier algorithme consiste à séparer et extraire les canaux RGB d'une image 2D et d'en faire une image 3D. Le principe réside dans le filtrage des couleurs par l'oeil devant lequel se trouve un filtre.
Chaque oeil ayant un filtre différent, il ne perçoit que les couleurs que le filtre laisse passer. Actuellement, le filtre le plus répandu est le filtre rouge/cyan.
Il suffit donc dans le cas de cet algortihme, d'extraire le canal rouge de l'image et d'en génerer un nouvelle. Ensuite il faut à partir de l'image source créer une nouvelle image à partir du mélange des canaux bleu et vert.
On obtient ainsi deux images, une cyan et l'autre rouge. Il faut ensuites les superposer en leur appliquant un décalage.\newline

La méthode de Dubois propose quant à elle de modifier les couleurs de l'image avant de lui appliquer l'algorithme vu précedemment.
En effet, afin de génerer des couleurs anaglyphe les plus proche possible de l'originale, Dubois tiens compte de la sensibilité spectrale de l'oeil humain,
le spectre d'absorption des filtre des lunettes et de la densité du spectre des moniteurs.
Dubois a alors pu en déduire une matrice à appliquer sur l'image originale permettant un effet 3D amélioré et la quasi
disparition des images fantômes qui sont des images résultat d'une supersition mal effectuée.
Le logiciel open-source Gimpel3D permet de faire de l'anaglyphe à partir d'images seulement et ne propose pas la méthode de Dubois. Nombre de logiciels payant existent et propose de réaliser des vidéos 3D
anaglyphe seuivant l'algortithme basique comme par exemple DVDFab 9 ou  3D Video Converter.

En ce qui concerne le side-by-side, le principe est de reproduire la distinction entre l'oeil droit et l'oeil gauche à travers une vidéo en comportant deux légérement décalés.
Il existe à l'heure actuelle beaucoup de logiciels qui proposent de faire ceci mais uniquement à partir de deux vidéos enregistrés préalablement avec le décalage.
Un seul logiciel exécute ce traitement avec une seule vidéo et il s'agit de DVDFab 9. Il est cependant payant et il n'existe aucun autre logiciel gratuit ou open-source réalisant ce traitement.

\section{Cahier des charges et architecture}

Besoins fonctionnels :\newline
\begin{itemize}
\item S'affiche sous forme de fenêtre.
\item Choisir le fichier à traiter.
\item Préciser si la vidéo est décodable par le logiciel ou non. Si non, précisez les formats acceptés.
\item choix du mode de traitement et de l'algorithme (pour l'anaglyphe choisir entre l'agorithme classique ou la méthode
de Dubois).
\item Effectuer le traitement à l'aide d'un bouton.
\item Montrer l'avancement du traitement.
\item Sauvegarder le fichier sous le nom que l'on veut.
\item ouvrir le fichier en fin de traitement ou son emplacement.
\item prévisualiser la vidéo séléctionnée\newline
\end{itemize}

Non fonctionnels :\newline
\begin{itemize}
\item Le logiciel doit être rapide. Le traitement doit s'effectuer dans le pire des cas dans le temps de la vidéo.
\item Le logiciel doit être simple à prendre en main et ne doit pas dérouter l'utilisateur.
\item Fidélité. Le logiciel ne doit pas détériorer la qualité de la vidéo originelle.
pdf.
\item Portable, le logiciel doit contenir toutes les biliothèques nécessaires pour fonctionner sur n'import quelle machine.\newline
\end{itemize}

Afin de traiter les flux vidéos et de faire le traitement nécessaire sur les images tirées des vidéos,
nous utiliserons la bibliothèque OpenImaj car multimédia, et maintenue à jour.
Cette bibliothèque présente de nombreux avantages. Tout d'abord elle est
disponible via un dépot Maven sous forme de modules. On peut donc récupérer uniquement les élèments de la bibliothèque
les plus pertinents. Nous aurons donc besoin ici des modules de décodage vidéo et de son, du module de traitemant d'images
et du module gérant les entrées/sorties.

OpenImaj présente également l'avantage d'être un wrapper autour de ffmpeg.
Cela simplifie donc grandement l'utilisation des traitements à faire d'autant que ffmpeg fait exactement ce que l'on veut ici,
c'est-à-dire extraire des images d'un flux vidéo.
On aura donc l'avantage ici de pouvoir décoder les formats vidéos
les plus utilisés mais aussi les fomats inhabituels, causant alors très peu de problèmes de compatibilité dans notre
programme.

L'encodage quant à lui est par contre beaucoup plus restreint. Cela ne nous empêchera pas de produire
le fichier de sortie dans les formats les plus courants c'est-à-dire MP4, AVI ou MKV.
En ce qui concerne l'image, notre programme donnera en sortie une image au format PNG afin d'éviter la perte de donnée et
pour pouvoir l'exporter facilement sur le web.
Le son des vidéos sera quant à lui restitué tel qu'il l'était sur le fichier source.


\section{Implémentation}
\section{Phase de test}

\section{Conclusion}







\end{document}
